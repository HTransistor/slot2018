%!TeX encoding=utf8
\input{preamble.tex}
\begin{document}
\title{\colorcap[luh_blue]{Scanning Laser Optical Tomography} \\ \grpnr $ $ (\Versuchsnummer)}
\subtitle{Laborpraktikum durchgeführt im Block 1\\
22.10.2018 – 09.11.2018
\vspace{1cm}\\
 \includegraphics[width=.75\linewidth]{IMAGE/luh_logo.png}}
\author{
\authA\\
\matA
\and
\authB\\
\matB
}
\date{\today}
%luh_logo blue: R:0, G: 80, B: 155, A:255 or Hex:00509b
%\definecolor{luh_blue}{RGB}{0,80,155} %in preamble

\pagestyle{empty} %Clear headline and footer
\setcounter{page}{0} %Set pagenumber to 0
\maketitle %Create the title

\newpage 

\thispagestyle{empty}
\tableofcontents
\pagestyle{scrheadings}

\setcounter{page}{1}
\newpage


\section{SLOT - Auflösungsvermögen}

Der Kontrast in Abhängigkeit von der horizontalen und vertikalen Auflösung wurde für verschiedene Fokussierungen, Strahlendurchmesser und Wellenlängen des Lasers gemessen. Dadurch ergibt sich die Modulationsübertragungsfunktion.
 
Bei allen Messungen wurde die Dunkelaufnahme von der normalen abgezogen, um ein Rauschen herauszurechnen.\\

Für geringe Auflösung wurde die Modulationsübertragungsfunktion  in Abhängikeit zur horizontalen Auflösung in Abbildung \ref{fig:Versuch2_Plot2h1} und zur vertikalen Auflösung in Abbildung \ref{fig:Versuch2_Plot2v1} dargestellt.\\
Für hohe Auflösung ist diese zur horizontalen Auflösung in Abbildung \ref{fig:Versuch2_Plot2h2} und zur vertikalen Auflösung in Abbildung \ref{fig:Versuch2_Plot2v2} dargestellt.

Außerdem sind die Modulationsübertragungsfunktionen für hohe Auflösungen in Abbildung \ref{fig:Versuch2_Plot2_all} dargestellt, um diese untereinander besser vergleichen zu können.



\begin{minipage}{\linewidth}
\begin{figure}[H]
	\centering
\includegraphics[width=1.0\linewidth]{IMAGE/Versuch2Plot1horizontal2.pdf}
	\caption{Kontrast bei geringer Auflösung (horizontal)}
	\label{fig:Versuch2_Plot2h1}
\end{figure} 

\begin{figure}[H]
	\centering
\includegraphics[width=1.0\linewidth]{IMAGE/Versuch2Plot1vertikal2.pdf}
	\caption{Kontrast bei geringer Auflösung (vertikal)}
	\label{fig:Versuch2_Plot2v1}
\end{figure} 

Ein erhöhter Strahlendurchmesser verbessert den Kontrast.\\
Bei geringer Auflösung ist der Kontrast für eine höhere Wellenlänge höher.\\
Der Kontrast ist in horizontaler und vertikaler Richtung von vergleichbarer Höhe.\\
Der Kontrast sinkt rapide, falls ohne Fokussierung gemessen wird.

\end{minipage}

\begin{minipage}{\linewidth}
\begin{figure}[H]
	\centering
\includegraphics[width=1.0\linewidth]{IMAGE/Versuch2Plot2horizontal2.pdf}
	\caption{Kontrast bei hoher Auflösung (horizontal)}
	\label{fig:Versuch2_Plot2h2}
\end{figure} 

\begin{figure}[H]
	\centering
\includegraphics[width=1.0\linewidth]{IMAGE/Versuch2Plot2vertikal2.pdf}
	\caption{Kontrast bei hoher Auflösung (vertikal)}
	\label{fig:Versuch2_Plot2v2}
\end{figure} 

Ein erhöhter Strahlendurchmesser verbessert den Kontrast. Bei hoher Auflösung ist der Kontrast für eine höhere Wellenlänge höher. Etwa bei einer Auflösung von $20 \frac{lp}{mm}$ erhöht eine kleine Wellenlänge den Kontrast.
\end{minipage}



\begin{figure}[H]
\centering
\includegraphics[width=1.0\linewidth]{IMAGE/Versuch2Plot2_all.pdf}
	\caption{Kontrast bei hoher Auflösung (horizontal \& vertikal)}
	\label{fig:Versuch2_Plot2_all}
\end{figure}

\textcolor{red}{Vielleicht sollte man die Innere mit der Äußeren Aufnahme zusammensetzen und im selben Plot anzeigen, um längere Messkurven zu erhalten.}


%%%%%%%%%%%%%%%%%%%%%%%Ab hier alte Vorlage%%%%%%%%%%%%%%%%%%%%%%%%%%%%%%%%%%%%
\section{Aufnahmen von Heuschreckengehirnen}
In diesem Versuchsabschitt haben wir als Probe ein präpariertes Heuschreckengehirn in den SLOT-Aufbau gestellt.
Die Probe war in einer Küvette und hat sich mit dieser um die $z$-Achse gedreht.
Mit einer 360°-Drehung um die $z$-Achse haben wir also unsere Aufnahmen unter verschieden Einstellungen aufgenommen.
Interessant war bei dieser Messung, dass wir die Fluoreszenz mit dem Photomultiplier (PMT) darstellen konnten, da es sich hier, im Gegensatz zum USAF-Target, um eine 3-dimensionale Probe handelt.

Wichtig für die Auswertung dieser Messung war, dass die Drehachse der Probe nicht präzessiert, was wir leider durch ausporbieren am Aufbau nicht ganz vermeiden konnten.
Die Achse wandert also ca. 4 Pixel von links nach rechts.

Die durchschnitttliche Schieflage $\alpha$  konnten wir bei der Auswertung jedoch ausgleichen.
Dazu haben wir den Drehachsendurchlauf am oberen und unteren Bildrand gemessen und den Winkel $\alpha$ mit
$$\alpha = \arcsin \left( \frac{\Delta{x_{\text{oben}}} - \Delta{x_{\text{unten}}}}{y_{\text{max}}} \right)$$
berechnet.
Hier entspricht $y_\text{max}$ der Anzahl an Pixel auf der $y$-Achse im Bild.

Um $\Delta{x_{\text{oben}}}$ und $\Delta{x_{\text{unten}}}$ messen zu können haben wir die jeweilige Ebene mit 100 verschiedenen $x$-Achsenverschiebungen mit ,,tilt'' rekonstruiert und anschließend das beste Bild herausgesucht (möglichst keine Ringartefakte).

Dann haben wir die Aufnahme mit ,,ImageJ'' um den jeweiligen Winkel gedreht und das Ergebnis wieder mit ,,tilt'' aber dieses mal für alle Ebenen rekonstruiert.
Dabei haben wir auch die mittlere $x$-Achsenverschiebung
$$\Delta{x} = \frac{\Delta{x_{\text{oben}}} - \Delta{x_{\text{unten}}}}{2}$$
beachtet.


Eine Verlängerung der Integrationszeit $\Delta{t}$ von $1 \si{s}$ zu $2 \si{s}$ hatte nur eine Aufhellung des PMT-Bildes zu Folge.
Da wir bei dieser Aufnahme jedoch einen unpassenden Filter verwendeten, zeigt das PMT-Bild auch nur das gestreute Licht. Das Ergebnis ist in Abbildung \ref{fig:lang-int} zu sehen.

\begin{figure}[ht]
\centering
\includegraphics[width=\linewidth]{IMAGE/both-2-450-5-1.png}
\caption{Rekonstruktion: Photodiode links, PMT rechts; $\lambda = 450 \si{nm}$, $\Delta{t} = 2 \si{s}$, $\lambda_\text{Filter} = (520 \pm 36) \si{nm}$, $d_\text{Strahl} = 5 \si{mm}$}
	\label{fig:lang-int}
\end{figure}

Mit der Photodiode lassen sich die äußeren Umrisse gut erkennen, mit dem PMT auch die Dichte im Inneren.

In Abbildung \ref{fig:both-pht} sind 2 Rekonstruktionen der Bilder der Photodiode dargestellt.
Man erkennt, dass das Bild mit der längeren Wellenlänge feinere Strukturen im Heuschreckengehirn darstellt.

\begin{figure}[ht]
\centering
\includegraphics[width=\linewidth]{IMAGE/2-pht.png}
\caption{Rekonstruktion: $\lambda = 520 \si{nm}$ links, $\lambda = 450 \si{nm}$ rechts; Photodiode, $\Delta{t} = 1 \si{s}$, $\lambda_\text{Filter} = (520 \pm 36) \si{nm}$, $d_\text{Strahl} = 5 \si{mm}$}
	\label{fig:both-pht}
\end{figure}

In Abbildung \ref{fig:both-pmt} wurden Bilder des PMT vom jeweiligen Fluoreszenzlicht der Laser rekonstruiert. Im Vergleich stellt man fest, dass man bei $\lambda= 520 \si{nm}$ mehr erkennt.

\begin{figure}[ht]
\centering
\includegraphics[width=\linewidth]{IMAGE/2-pmt.png}
\caption{Rekonstruktion: $\lambda = 520 \si{nm}$ links mit $\lambda_\text{Filter} = (676 \pm 29) \si{nm}$ links, $\lambda = 450 \si{nm}$ mit $\lambda_\text{Filter} \ge 570 \si{nm}$ rechts; Photodiode, $\Delta{t} = 1 \si{s}$, $d_\text{Strahl} = 5 \si{mm}$}
	\label{fig:both-pmt}
\end{figure}



\clearpage
\begin{thebibliography}{99}
	\bibitem{Schaltung1} \textsc{Saure aus Wikimedia Commons}, \emph{Gleichrichter-Schaltung mit Glättung} (26. August 2009) (Stand: 08.03.2018) \url{https://commons.wikimedia.org/w/index.php?title=File:Gleichrichter-Schaltung.svg&oldid=291347227&uselang=de}
\end{thebibliography}

\end{document}
