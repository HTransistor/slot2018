%!TeX encoding=utf8
\input{preamble.tex}
\begin{document}
\title{\colorcap[luh_blue]{Scanning Laser Optical Tomography} \\ \grpnr $ $ (\Versuchsnummer)}
\subtitle{Laborpraktikum durchgeführt im Block 1\\
22.10.2018 – 09.11.2018
\vspace{1cm}\\
 \includegraphics[width=.75\linewidth]{IMAGE/luh_logo.png}}
\author{
\authA\\
\matA
\and
\authB\\
\matB
}
\date{\today}
%luh_logo blue: R:0, G: 80, B: 155, A:255 or Hex:00509b
%\definecolor{luh_blue}{RGB}{0,80,155} %in preamble

\pagestyle{empty} %Clear headline and footer
\setcounter{page}{0} %Set pagenumber to 0
\maketitle %Create the title

\newpage 

\thispagestyle{empty}
\tableofcontents
\pagestyle{scrheadings}

\setcounter{page}{1}
\newpage


\section{Einleitung}
In diesem Versuch wird die Methode der Tomographie am Beispiel SLOT \\ \glqq \colorcap[luh_blue]{Scanning Laser Optical Tomography} \grqq untersucht. Dieses ist ein spezielles dreidimensionales Tomographie-Verfahren, das mit Lasern verschiedener Wellenlängen mittels der Ausnutzung von Streuung, Absorption und Fluoreszenz ein bildgebendes Verfahren liefert.\\

Das dreidimensionale Präparat wird ebenenweise durch zweidimensionale Schnitte (Tomogramme) konstruiert und zu einem dreidimensionalen Bildobjekt zusammengesetzt. Das entstandene Bildobjekt ist nun vom Präparat getrennt und virtuell verfügbar. Daraus ergibt sich der Vorteil das Präparat unbeschädigt lassen zu können und dennoch die innere Struktur des Präparates zu untersuchen \cite{Anleitung}.\\

\section{Scanning Laser Optical Tomography}

\subsection{Radontransformation}
Bei allen Computertomographie-Verfahren ist die Radontransformation die mathematische Grundlage, um das integrierte Signal eines Strahls durch das Präparat für eine Rekonstruktion zu nutzen.

Die Strahlen werden hierfür in Zylinderkoordinaten durch eine Schar mit Laufparameter s und parametrisiert:
\begin{equation}
lin_{p,q}(s) = \begin{pmatrix}
\cos(\phi) & -\sin(\phi)  \\
\sin(\phi) & \cos(\phi) \\
\end{pmatrix}
\cdot
\begin{pmatrix}
\rho \\
s \\
\end{pmatrix}
 =
\begin{pmatrix}
\rho \cdot \cos(\phi) - s \cdot \sin(\phi) \\
\rho \cdot \sin(\phi) + s \cdot \cos(\phi) \\
\end{pmatrix}
\end{equation}

%\begin{minipage}{\linewidth}
\begin{figure}[H]
	\centering
\includegraphics[width=0.6\linewidth]{IMAGE/radon.png}
	\caption{Graphische Darstellung der Radontransformation \cite{slot_paper}}
	\label{fig:radon}
\end{figure}

Für eine zweidimensionale Funktion $f(x,y)$ ist nun die Radontransformierte das integrierte Signal über den parametrisierten Weg durch das Präparat:

\begin{equation}
R(\rho,\phi) = \int_{- \infty}^{\infty} f(lin_{p,q}(s)) \,\mathrm{d}s
=
\int_{- \infty}^{\infty} f\begin{pmatrix}
\rho \cdot \cos(\phi) - s \cdot \sin(\phi) \\
\rho \cdot \sin(\phi) + s \cdot \cos(\phi) \\
\end{pmatrix} \,\mathrm{d}s
\end{equation}

Entscheidend für die Nutzung bei der Tomographie ist die Invertierbarkeit der Radontransformierten $R(\rho,\phi)$, denn dies ermöglicht die Rekonstruktion der ursprünglichen Funktion $f(x,y)$, also die Rekonstruktion des Präparates \cite{slot_paper}.
%\end{minipage}

\subsection{Funktionsweise}
\begin{minipage}{\linewidth}
SLOT basiert auf einem x-y-Scanner-System mit zwei Silberspiegeln, das den auf die Mitte des Präparates fokussierten Laser abläuft. Das Präparat muss \glqq aufgeklart\grqq sein, um Streuung beim Übergang zwischen Präparat und Träger des Präparats zu verringern und eine höhere Transmission zu erreichen. Hierzu wird Wasser oder Glycerin genutzt, da dieses einen ähnlichen Brechungsindex wie organische Präparate haben. Das transmittierte Laserlicht wird von einer Photodiode hinter der Probenkammer detektiert. Um eine Rekonstruktion durchführen zu können sind Aufnahmen des Präparats von verschiedenen Richtungen aus nötig, Hintergrund ist die Radontransformation. Diese verschiedenen Aufnahmen werden durch einen Motor ermöglicht der die Kapillare und damit das Präparat dreht. Weiterhin regt das Laserlicht das Präparat zur Fluoreszenz an, welches mittels plankovexen Linsen mit dazwischen liegendem Fluoreszenz-Filter gesammelt und auf den sensitiven Photomuliplier (PMT) geleitet wird \cite{Anleitung}.\\

\begin{figure}[H]
	\centering
\includegraphics[width=0.6\linewidth]{IMAGE/slot_schema.png}
	\caption{Schema eines scannenden laseroptischen Tomographen (SLOT) \cite{slot_paper}}
	\label{fig:schema}
\end{figure} 
\end{minipage}

%%%nopagebreak (gehört zusammen)
 \nopagebreak
 \hrule height 0pt
 \pagebreak[2]
%%% 

\subsection{Versuchsaufbau}

\begin{minipage}{\linewidth}
\begin{figure}[H]
	\centering
\includegraphics[width=1.0\linewidth]{IMAGE/versuchsaufbau.png}
	\caption{Versuchsaufbau \cite{Anleitung}}
	\label{fig:versuchsaufbau}
\end{figure} 

\begin{figure}[H]
	\centering
\includegraphics[width=1.0\linewidth]{IMAGE/scanner.jpeg}
	\caption{Teleskop und Scanner \cite{Anleitung}}
	\label{fig:scanner}
\end{figure}

\end{minipage}

\begin{figure}
\begin{subfigure}[b]{0.5\linewidth}
\includegraphics[width=0.98\linewidth]{IMAGE/einkopplung.jpeg}
\label{fig:einkopplung}
\subcaption{Einkopplung der Laserdiode}

\end{subfigure}
\begin{subfigure}[b]{0.5\linewidth}
\includegraphics[width=0.98\linewidth]{IMAGE/pmt.png}\label{fig:pmt}
\subcaption[c]{Filterrad und Photomuliplier (PMT)}
\end{subfigure}

\caption{Trennung der Laserkonfiguration vom restlichen Versuchsaufbau \cite{Anleitung}}
\end{figure}
%%%%%%%%%%%%%%%
\clearpage

\begin{minipage}{\linewidth}
\begin{figure}[H]
	\centering
\includegraphics[width=0.491\linewidth]{IMAGE/turm.png}
	\caption{Turm mit Rotationsachse \cite{Anleitung}}
	\label{fig:turm}
\end{figure}

\section{Ergebnisse}
\subsection{Auflösungsvermögen}
Der Kontrast in Abhängigkeit von der horizontalen und vertikalen Auflösung wurde für verschiedene Fokussierungen, Strahlendurchmesser und Wellenlängen des Lasers gemessen. Dadurch ergibt sich die Modulationsübertragungsfunktion.\\ 
Bei allen Messungen wurde die Dunkelaufnahme von der normalen abgezogen, um ein Rauschen herauszurechnen.\\
Für geringe Auflösung wurde die Modulationsübertragungsfunktion  in Abhängikeit zur horizontalen Auflösung in Abbildung \ref{fig:Versuch2_Plot2h1} und zur vertikalen Auflösung in Abbildung \ref{fig:Versuch2_Plot2v1} dargestellt.\\
\end{minipage}

\begin{minipage}{\linewidth}
\begin{figure}[H]
	\centering
\includegraphics[width=1.0\linewidth]{IMAGE/Versuch2Plot1horizontal2.pdf}
	\caption{Kontrast bei geringer Auflösung (horizontal)}
	\label{fig:Versuch2_Plot2h1}
\end{figure} 

\begin{figure}[H]
	\centering
\includegraphics[width=1.0\linewidth]{IMAGE/Versuch2Plot1vertikal2.pdf}
	\caption{Kontrast bei geringer Auflösung (vertikal)}
	\label{fig:Versuch2_Plot2v1}
\end{figure} 

Ein erhöhter Strahlendurchmesser verbessert den Kontrast.\\
Bei geringer Auflösung ist der Kontrast für eine höhere Wellenlänge höher.\\
Der Kontrast ist in horizontaler und vertikaler Richtung von vergleichbarer Höhe.\\
Weiterhin sinkt der Kontrast rapide, falls ohne Fokussierung gemessen wird.
\end{minipage}

\begin{minipage}{\linewidth}
Für hohe Auflösung ist diese zur horizontalen Auflösung in Abbildung \ref{fig:Versuch2_Plot2h2} und zur vertikalen Auflösung in Abbildung \ref{fig:Versuch2_Plot2v2} dargestellt.
\begin{figure}[H]
	\centering
\includegraphics[width=1.0\linewidth]{IMAGE/Versuch2Plot2horizontal2.pdf}
	\caption{Kontrast bei hoher Auflösung (horizontal)}
	\label{fig:Versuch2_Plot2h2}
\end{figure} 

\begin{figure}[H]
	\centering
\includegraphics[width=1.0\linewidth]{IMAGE/Versuch2Plot2vertikal2.pdf}
	\caption{Kontrast bei hoher Auflösung (vertikal)}
	\label{fig:Versuch2_Plot2v2}
\end{figure} 
Auch bei hoher Auflösung verbessert ein erhöhter Strahlendurchmesser  den Kontrast. Zwar ist der Kontrast bei einer geringen Auflösung bei einer höheren Wellenlänge höher, aber ab etwa einer Auflösung von $20 \frac{lp}{mm}$ erhöht eine kleine Wellenlänge den Kontrast.
\end{minipage}

Weiterhin sind die Modulationsübertragungsfunktionen für hohe Auflösungen in Abbildung \ref{fig:Versuch2_Plot2_all} dargestellt, um diese untereinander besser vergleichen zu können.

\begin{figure}[H]
\centering
\includegraphics[width=1.0\linewidth]{IMAGE/Versuch2Plot2_all.pdf}
	\caption{Kontrast bei hoher Auflösung (horizontal \& vertikal)}
	\label{fig:Versuch2_Plot2_all}
\end{figure}

%\textcolor{red}{Vielleicht sollte man die Innere mit der Äußeren Aufnahme zusammensetzen und im selben Plot anzeigen, um längere Messkurven zu erhalten.}


%%%%%%%%%%%%%%%%%%%%%%%Ab hier alte Vorlage%%%%%%%%%%%%%%%%%%%%%%%%%%%%%%%%%%%%
\subsection{Aufnahmen von Heuschreckengehirnen}
In diesem Versuchsabschitt wurde als Probe ein präpariertes Heuschreckengehirn gewählt.
Die Probe wurde in einer Küvette von oben in Glycerin getaucht und wurde um die $z$-Achse gedreht.
Mit einer 360°-Drehung um die $z$-Achse wurden die Aufnahmen unter verschiedenen Winkeln aufgenommen.
Interessant war bei dieser Messung, dass die Fluoreszenz mit dem Photomultiplier (PMT) darstellt werden konnte, da es sich hier, im Gegensatz zum USAF-Target, um eine dreidimensionale Probe handelt, die fluoresziert.

\subsubsection{Korrektur der Schieflage}
Die durchschnittliche Schieflage der Probe $\alpha$ konnte bei der Auswertung ausgeglichen werden.
Dazu wurde der Drehachsendurchlauf am oberen und unteren Bildrand gemessen und der Winkel $\alpha$ nach folgender Formel berechnet:
$$\alpha = \arcsin \left( \frac{\Delta{x_{\text{oben}}} - \Delta{x_{\text{unten}}}}{y_{\text{max}}} \right)$$

Hier entspricht $y_\text{max}$ der Anzahl der Pixel auf der $y$-Achse im Bild.
$\Delta{x_\text{oben}}$ und $\Delta{x_\text{unten}}$ beziehen sich hierbei auf die äußersten Pixel am Bildrand oben und unten.

Um $\Delta{x_{\text{oben}}}$ und $\Delta{x_{\text{unten}}}$ zu messen wurde die jeweilige Ebene mit 100 verschiedenen $x$-Achsenverschiebungen mit  \glqq tilt\grqq rekonstruiert und anschließend das beste Bild herausgesucht (Reduzierung der Ringartefakte).

Im Folgenden wurde die Aufnahme mit \glqq ImageJ\grqq um den jeweiligen Winkel gedreht und das Ergebnis wieder mit  \glqq tilt\grqq , aber dieses mal für alle Ebenen rekonstruiert.
Dabei wurde auch die mittlere $x$-Achsenverschiebung beachtet:
$$\Delta{x} = \frac{\Delta{x_{\text{oben}}} - \Delta{x_{\text{unten}}}}{2}$$

\subsubsection{Darstellung der Rekonstruktion}
Eine Verlängerung der Integrationszeit $\Delta{t}$ von $1$ zu $2$ Sekunden hatte nur eine Aufhellung des PMT-Bildes zur Folge.
Da bei dieser Aufnahme jedoch ein unpassender Filter verwendetet wurde, zeigt das PMT-Bild auch nur das gestreute Licht. Das Ergebnis ist in Abbildung \ref{fig:lang-int} zu sehen.

\begin{figure}[ht]
\centering
\includegraphics[width=\linewidth]{IMAGE/both-2-450-5-1-c.png}
\caption{Rekonstruktion: Photodiode links, PMT rechts; $\lambda = 450 \si{nm}$, $\Delta{t} = 2 \si{s}$, $\lambda_\text{Filter} = (520 \pm 36) \si{nm}$, $d_\text{Strahl} = 5 \si{mm}$}
	\label{fig:lang-int}
\end{figure}

Mit der Photodiode lassen sich die äußeren Umrisse gut erkennen, mit dem Photomultiplier sogar die Dichte im Inneren.

\begin{minipage}{\linewidth}
In Abbildung \ref{fig:both-pht} sind 2 Rekonstruktionen der Bilder der Photodiode dargestellt.
Es ist erkennbar, dass das Bild mit der längeren Wellenlänge feinere Strukturen im Heuschreckengehirn auflöst.
\begin{figure}[H]
\centering
\includegraphics[width=\linewidth]{IMAGE/2-pht-c.png}
\caption{Rekonstruktion: $\lambda = 520 \si{nm}$ links, $\lambda = 450 \si{nm}$ rechts; Photodiode, $\Delta{t} = 1 \si{s}$, $\lambda_\text{Filter} = (520 \pm 36) \si{nm}$, $d_\text{Strahl} = 5 \si{mm}$}
	\label{fig:both-pht}
\end{figure}

In Abbildung \ref{fig:both-pmt} wurden Bilder des PMT vom jeweiligen Fluoreszenzlicht der Laser rekonstruiert. Im Vergleich ist festzustellen, dass bei $\lambda= 520 \si{nm}$ die Auflösung höher ist.

\begin{figure}[H]
\centering
\includegraphics[width=\linewidth]{IMAGE/2-pmt-c.png}
\caption{Rekonstruktion: $\lambda = 520 \si{nm}$ links mit $\lambda_\text{Filter} = (676 \pm 29) \si{nm}$ links,\\
$\lambda = 450 \si{nm}$ mit $\lambda_\text{Filter} \ge 570 \si{nm}$ rechts; Photodiode, $\Delta{t} = 1 \si{s}$,\\
$d_\text{Strahl} = 5 \si{mm}$}
	\label{fig:both-pmt}
\end{figure}
\end{minipage}

\begin{minipage}{\linewidth}
Mit dem ImageJ-Plugin \glqq Volume Viewer\grqq \, ist es nach der Rekonstruktion möglich verschiedene Ansichten auf das Heuschreckengehirn zu generieren. Ein mögliche Ansicht ist in Abbildung \ref{fig:3d} dargestellt.

\begin{figure}[H]
\centering
\includegraphics[width=\linewidth]{IMAGE/3dtomo.png}
\caption{3-dimensionale Ansicht einer PMT-Aufnahme: $\lambda = 520 \si{nm}$ mit\\ $\lambda_\text{Filter} = (676 \pm 29) \si{nm}$, $\Delta{t} = 1 \si{s}$, $d_\text{Strahl} = 5 \si{mm}$}
	\label{fig:3d}
\end{figure}

\subsubsection{Analyse}
Wichtig für die Auswertung dieser Messung war, dass die Drehachse der Probe nicht präzessiert, was durch die Justage am Aufbau nicht ganz vermieden werden konnte.
Die Achse wandert folglich circa 4 Pixel in horizontaler Richtung.

\end{minipage}

%\listoftables
\clearpage
\listoffigures
%\clearpage
\begin{thebibliography}{99}
%	\bibitem{Schaltung1} \textsc{Saure aus Wikimedia Commons}, \emph{Gleichrichter-Schaltung mit Glättung} (26. August 2009) (Stand: 08.03.2018) \url{https://commons.wikimedia.org/w/index.php?title=File:Gleichrichter-Schaltung.svg&oldid=291347227&uselang=de}
\bibitem{Anleitung} \textsc{Lena Nolte}, \emph{Versuchsanleitung: IQ18 SLOT für das Laborpraktikum Atom- und Molekühlphysik
der Leibniz Universität Hannover} (2015) 

\bibitem{slot_paper} \textsc{Raoul-Amadeus Lorbeer}, \emph{Dreidimensionale und effiziente Erfassung mesoskopischer Proben} (2013) 
\end{thebibliography}



\end{document}
